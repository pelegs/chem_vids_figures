\documentclass[border=1cm]{standalone}
\usepackage{chemfig,chemmacros}

% Make chemfig prettier
\setchemfig{
	fixed length=false,
	double bond sep=0.35700em,  % 'Bond Spacing'
	bond style={line width=0.06642em},
	atomsep=1.78500em,    % 'Fixed Length'
}
\makeatletter
\def\Hv@scale{.95}
\makeatother
\DeclareMathAlphabet{\foo}{OT1}{phv}{m}{n}
\renewcommand*\printatom[1]{\ensuremath{\foo{#1}}}

% Colors
\definecolor{xgreen}{HTML}{52B256}
\colorlet{darkgreen}{xgreen!90!black}

% Doc
\begin{document}
\begin{centering}
\schemestart
\chemname[5mm]{\chemfig{-[:30]-[:330]-[:30]-[:330]\color{red}{O}H}}{Butanol}
\+
\chemname[5mm]{\chemfig{-[:30](=[:90]\color{darkgreen}{O})-[:330]\color{darkgreen}{O}-[:30](-[:330])=[:90]\color{blue}{O}}}{Acetic anhydride}
\arrow{->[\chemfig{CH_{3}COOH}][(15\%\,\chemfig{H_{2}SO_{4}})][][]}[0, 1.6]
\chemname[5mm]{\chemfig{-[:30]([:90]=\color{blue}{O})-[:330]\color{red}{O}-[:30]-[:330]-[:30]-[:330]}}{Butyl acetate}
\+
\chemname[5mm]{\chemfig{-[:30]([:90]=\color{darkgreen}{O})-[:330]\color{darkgreen}{O}H}}{Acetic acid}
\schemestop
\end{centering}
\end{document}
